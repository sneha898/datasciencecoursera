% Options for packages loaded elsewhere
\PassOptionsToPackage{unicode}{hyperref}
\PassOptionsToPackage{hyphens}{url}
%
\documentclass[
]{article}
\usepackage{lmodern}
\usepackage{amssymb,amsmath}
\usepackage{ifxetex,ifluatex}
\ifnum 0\ifxetex 1\fi\ifluatex 1\fi=0 % if pdftex
  \usepackage[T1]{fontenc}
  \usepackage[utf8]{inputenc}
  \usepackage{textcomp} % provide euro and other symbols
\else % if luatex or xetex
  \usepackage{unicode-math}
  \defaultfontfeatures{Scale=MatchLowercase}
  \defaultfontfeatures[\rmfamily]{Ligatures=TeX,Scale=1}
\fi
% Use upquote if available, for straight quotes in verbatim environments
\IfFileExists{upquote.sty}{\usepackage{upquote}}{}
\IfFileExists{microtype.sty}{% use microtype if available
  \usepackage[]{microtype}
  \UseMicrotypeSet[protrusion]{basicmath} % disable protrusion for tt fonts
}{}
\makeatletter
\@ifundefined{KOMAClassName}{% if non-KOMA class
  \IfFileExists{parskip.sty}{%
    \usepackage{parskip}
  }{% else
    \setlength{\parindent}{0pt}
    \setlength{\parskip}{6pt plus 2pt minus 1pt}}
}{% if KOMA class
  \KOMAoptions{parskip=half}}
\makeatother
\usepackage{xcolor}
\IfFileExists{xurl.sty}{\usepackage{xurl}}{} % add URL line breaks if available
\IfFileExists{bookmark.sty}{\usepackage{bookmark}}{\usepackage{hyperref}}
\hypersetup{
  pdftitle={CProj1},
  pdfauthor={Sneha Bhattacharjee},
  hidelinks,
  pdfcreator={LaTeX via pandoc}}
\urlstyle{same} % disable monospaced font for URLs
\usepackage[margin=1in]{geometry}
\usepackage{color}
\usepackage{fancyvrb}
\newcommand{\VerbBar}{|}
\newcommand{\VERB}{\Verb[commandchars=\\\{\}]}
\DefineVerbatimEnvironment{Highlighting}{Verbatim}{commandchars=\\\{\}}
% Add ',fontsize=\small' for more characters per line
\usepackage{framed}
\definecolor{shadecolor}{RGB}{248,248,248}
\newenvironment{Shaded}{\begin{snugshade}}{\end{snugshade}}
\newcommand{\AlertTok}[1]{\textcolor[rgb]{0.94,0.16,0.16}{#1}}
\newcommand{\AnnotationTok}[1]{\textcolor[rgb]{0.56,0.35,0.01}{\textbf{\textit{#1}}}}
\newcommand{\AttributeTok}[1]{\textcolor[rgb]{0.77,0.63,0.00}{#1}}
\newcommand{\BaseNTok}[1]{\textcolor[rgb]{0.00,0.00,0.81}{#1}}
\newcommand{\BuiltInTok}[1]{#1}
\newcommand{\CharTok}[1]{\textcolor[rgb]{0.31,0.60,0.02}{#1}}
\newcommand{\CommentTok}[1]{\textcolor[rgb]{0.56,0.35,0.01}{\textit{#1}}}
\newcommand{\CommentVarTok}[1]{\textcolor[rgb]{0.56,0.35,0.01}{\textbf{\textit{#1}}}}
\newcommand{\ConstantTok}[1]{\textcolor[rgb]{0.00,0.00,0.00}{#1}}
\newcommand{\ControlFlowTok}[1]{\textcolor[rgb]{0.13,0.29,0.53}{\textbf{#1}}}
\newcommand{\DataTypeTok}[1]{\textcolor[rgb]{0.13,0.29,0.53}{#1}}
\newcommand{\DecValTok}[1]{\textcolor[rgb]{0.00,0.00,0.81}{#1}}
\newcommand{\DocumentationTok}[1]{\textcolor[rgb]{0.56,0.35,0.01}{\textbf{\textit{#1}}}}
\newcommand{\ErrorTok}[1]{\textcolor[rgb]{0.64,0.00,0.00}{\textbf{#1}}}
\newcommand{\ExtensionTok}[1]{#1}
\newcommand{\FloatTok}[1]{\textcolor[rgb]{0.00,0.00,0.81}{#1}}
\newcommand{\FunctionTok}[1]{\textcolor[rgb]{0.00,0.00,0.00}{#1}}
\newcommand{\ImportTok}[1]{#1}
\newcommand{\InformationTok}[1]{\textcolor[rgb]{0.56,0.35,0.01}{\textbf{\textit{#1}}}}
\newcommand{\KeywordTok}[1]{\textcolor[rgb]{0.13,0.29,0.53}{\textbf{#1}}}
\newcommand{\NormalTok}[1]{#1}
\newcommand{\OperatorTok}[1]{\textcolor[rgb]{0.81,0.36,0.00}{\textbf{#1}}}
\newcommand{\OtherTok}[1]{\textcolor[rgb]{0.56,0.35,0.01}{#1}}
\newcommand{\PreprocessorTok}[1]{\textcolor[rgb]{0.56,0.35,0.01}{\textit{#1}}}
\newcommand{\RegionMarkerTok}[1]{#1}
\newcommand{\SpecialCharTok}[1]{\textcolor[rgb]{0.00,0.00,0.00}{#1}}
\newcommand{\SpecialStringTok}[1]{\textcolor[rgb]{0.31,0.60,0.02}{#1}}
\newcommand{\StringTok}[1]{\textcolor[rgb]{0.31,0.60,0.02}{#1}}
\newcommand{\VariableTok}[1]{\textcolor[rgb]{0.00,0.00,0.00}{#1}}
\newcommand{\VerbatimStringTok}[1]{\textcolor[rgb]{0.31,0.60,0.02}{#1}}
\newcommand{\WarningTok}[1]{\textcolor[rgb]{0.56,0.35,0.01}{\textbf{\textit{#1}}}}
\usepackage{graphicx,grffile}
\makeatletter
\def\maxwidth{\ifdim\Gin@nat@width>\linewidth\linewidth\else\Gin@nat@width\fi}
\def\maxheight{\ifdim\Gin@nat@height>\textheight\textheight\else\Gin@nat@height\fi}
\makeatother
% Scale images if necessary, so that they will not overflow the page
% margins by default, and it is still possible to overwrite the defaults
% using explicit options in \includegraphics[width, height, ...]{}
\setkeys{Gin}{width=\maxwidth,height=\maxheight,keepaspectratio}
% Set default figure placement to htbp
\makeatletter
\def\fps@figure{htbp}
\makeatother
\setlength{\emergencystretch}{3em} % prevent overfull lines
\providecommand{\tightlist}{%
  \setlength{\itemsep}{0pt}\setlength{\parskip}{0pt}}
\setcounter{secnumdepth}{-\maxdimen} % remove section numbering

\title{CProj1}
\author{Sneha Bhattacharjee}
\date{10/17/2020}

\begin{document}
\maketitle

\hypertarget{introduction}{%
\subsubsection{Introduction}\label{introduction}}

In this peer graded assignment, i will be investigating the exponential
distribution in R and hence compare it with the Central Limit Theorem. I
will be investigating the distribution of averages of 40 exponentials. A
total of 1000 simulations will be done.

\hypertarget{setup}{%
\subsubsection{Setup}\label{setup}}

\hypertarget{load-packages-and-library}{%
\subsubsection{Load packages and
library}\label{load-packages-and-library}}

For the purpose of this analysis, i will be using the library `ggplot2'.

\begin{Shaded}
\begin{Highlighting}[]
\KeywordTok{library}\NormalTok{(ggplot2)}
\end{Highlighting}
\end{Shaded}

\hypertarget{illustration-of-simulations}{%
\subsubsection{Illustration of
simulations}\label{illustration-of-simulations}}

\begin{enumerate}
\def\labelenumi{\arabic{enumi}.}
\tightlist
\item
  Show the sample mean and compare it to theoretical mean of
  distribution obtained. As per given instructions, the exponential
  distributions can be simulated in R using rexp(n, lambda) where,
  lambda is rate parameter. The simulation is repeated 1000 times here.
  The theoretical mean of exponential distribution is 1/lambda and the
  standard deviation is also 1/lambda. Here, for 1000 simulations,
  lambda is considered 0.2 and sample size ,i.e, n is 40.
\end{enumerate}

\begin{Shaded}
\begin{Highlighting}[]
\CommentTok{#Setting the seed to ensure reproduceability}
\KeywordTok{set.seed}\NormalTok{(}\DecValTok{11081990}\NormalTok{)}

\CommentTok{#Setting lambda as 0.2}
\NormalTok{lambda <-}\StringTok{ }\FloatTok{0.2} 
\CommentTok{#Set values of n}
\NormalTok{n <-}\StringTok{ }\DecValTok{40}

\CommentTok{#Obtain the sample mean of 1 simulation with the above parameters}
\CommentTok{#Obtain sample mean of 1000 simulations with the same above parameters}
\CommentTok{#Calculate the mean of this simulation of sample means}
\NormalTok{expoDist <-}\StringTok{ }\KeywordTok{matrix}\NormalTok{(}\DataTypeTok{data=}\KeywordTok{rexp}\NormalTok{(n }\OperatorTok{*}\StringTok{ }\DecValTok{1000}\NormalTok{, lambda), }\DataTypeTok{nrow=}\DecValTok{1000}\NormalTok{)}
\NormalTok{expoDistMeans <-}\StringTok{ }\KeywordTok{data.frame}\NormalTok{(}\DataTypeTok{means=}\KeywordTok{apply}\NormalTok{(expoDist, }\DecValTok{1}\NormalTok{, mean))}

\CommentTok{#Plot the means obtained}
\KeywordTok{ggplot}\NormalTok{(}\DataTypeTok{data =}\NormalTok{ expoDistMeans, }\KeywordTok{aes}\NormalTok{(}\DataTypeTok{x =}\NormalTok{ means)) }\OperatorTok{+}\StringTok{ }
\StringTok{  }\KeywordTok{geom_histogram}\NormalTok{(}\DataTypeTok{binwidth=}\FloatTok{0.1}\NormalTok{) }\OperatorTok{+}\StringTok{  }\KeywordTok{stat_function}\NormalTok{(}\DataTypeTok{fun =}\NormalTok{ dnorm, }\DataTypeTok{color =} \StringTok{"orange"}\NormalTok{) }\OperatorTok{+}
\StringTok{  }\KeywordTok{scale_x_continuous}\NormalTok{(}\DataTypeTok{breaks=}\KeywordTok{round}\NormalTok{(}\KeywordTok{seq}\NormalTok{(}\KeywordTok{min}\NormalTok{(expoDistMeans}\OperatorTok{$}\NormalTok{means), }\KeywordTok{max}\NormalTok{(expoDistMeans}\OperatorTok{$}\NormalTok{means), }\DataTypeTok{by=}\DecValTok{1}\NormalTok{)))}
\end{Highlighting}
\end{Shaded}

\includegraphics{CourseProject11_files/figure-latex/simu1.1-1.pdf}

\#Theoretical mean that is the centre of distribution is

\begin{Shaded}
\begin{Highlighting}[]
\NormalTok{th <-}\StringTok{ }\DecValTok{1}\OperatorTok{/}\NormalTok{lambda}
\NormalTok{th  }\CommentTok{#Let Th be the expected mean }
\end{Highlighting}
\end{Shaded}

\begin{verbatim}
## [1] 5
\end{verbatim}

\begin{Shaded}
\begin{Highlighting}[]
\NormalTok{mean1000 <-}\StringTok{ }\KeywordTok{mean}\NormalTok{(expoDistMeans}\OperatorTok{$}\NormalTok{means)}
\NormalTok{mean1000}
\end{Highlighting}
\end{Shaded}

\begin{verbatim}
## [1] 5.002927
\end{verbatim}

The centre of distribution of sample means of 40 exponentials is close
to the theoretical center of the distribution.

\hypertarget{show-how-variable-the-sample-is-via-variance-and-compare-it-to-the-theoretical-variance-of-the-distribution.}{%
\subsubsection{2. Show how variable the sample is (via variance) and
compare it to the theoretical variance of the
distribution.}\label{show-how-variable-the-sample-is-via-variance-and-compare-it-to-the-theoretical-variance-of-the-distribution.}}

\begin{Shaded}
\begin{Highlighting}[]
\CommentTok{#Find the expected standard deviation }
\NormalTok{sd <-}\StringTok{ }\DecValTok{1}\OperatorTok{/}\NormalTok{lambda}\OperatorTok{/}\KeywordTok{sqrt}\NormalTok{(n)}
\NormalTok{sd}
\end{Highlighting}
\end{Shaded}

\begin{verbatim}
## [1] 0.7905694
\end{verbatim}

\begin{Shaded}
\begin{Highlighting}[]
\CommentTok{#Variance of standard deviation obtained as}
\NormalTok{Vare <-}\StringTok{ }\NormalTok{sd}\OperatorTok{^}\DecValTok{2}
\NormalTok{Vare}
\end{Highlighting}
\end{Shaded}

\begin{verbatim}
## [1] 0.625
\end{verbatim}

\begin{Shaded}
\begin{Highlighting}[]
\NormalTok{sd_x <-}\StringTok{ }\KeywordTok{sd}\NormalTok{(expoDistMeans}\OperatorTok{$}\NormalTok{means)}
\NormalTok{sd_x}
\end{Highlighting}
\end{Shaded}

\begin{verbatim}
## [1] 0.7767084
\end{verbatim}

\begin{Shaded}
\begin{Highlighting}[]
\NormalTok{Var_x <-}\StringTok{ }\KeywordTok{var}\NormalTok{(expoDistMeans}\OperatorTok{$}\NormalTok{means)}
\NormalTok{Var_x}
\end{Highlighting}
\end{Shaded}

\begin{verbatim}
## [1] 0.6032759
\end{verbatim}

We can observe that the standar deviations are very close in value.
Although minor differences exist as variance is the square of standard
deviation.

\hypertarget{to-show-that-distribution-is-approximately-normal.}{%
\subsubsection{3.To show that distribution is approximately
normal.}\label{to-show-that-distribution-is-approximately-normal.}}

\begin{Shaded}
\begin{Highlighting}[]
\CommentTok{#To plot the sample means obtained from the simulation.}
\KeywordTok{ggplot}\NormalTok{(}\DataTypeTok{data =}\NormalTok{ expoDistMeans, }\KeywordTok{aes}\NormalTok{(}\DataTypeTok{x =}\NormalTok{ means)) }\OperatorTok{+}\StringTok{ }
\StringTok{  }\KeywordTok{geom_histogram}\NormalTok{(}\DataTypeTok{binwidth=}\FloatTok{0.1}\NormalTok{, }\KeywordTok{aes}\NormalTok{(}\DataTypeTok{y=}\NormalTok{..density..), }\DataTypeTok{alpha=}\FloatTok{0.2}\NormalTok{) }\OperatorTok{+}\StringTok{ }
\StringTok{  }\KeywordTok{stat_function}\NormalTok{(}\DataTypeTok{fun =}\NormalTok{ dnorm, }\DataTypeTok{arg =} \KeywordTok{list}\NormalTok{(}\DataTypeTok{mean =}\NormalTok{ th , }\DataTypeTok{sd =}\NormalTok{ sd), }\DataTypeTok{colour =} \StringTok{"red"}\NormalTok{, }\DataTypeTok{size=}\DecValTok{1}\NormalTok{) }\OperatorTok{+}\StringTok{ }
\StringTok{  }\KeywordTok{geom_vline}\NormalTok{(}\DataTypeTok{xintercept =}\NormalTok{ th, }\DataTypeTok{size=}\DecValTok{1}\NormalTok{, }\DataTypeTok{colour=}\StringTok{"#CC0000"}\NormalTok{) }\OperatorTok{+}\StringTok{ }
\StringTok{  }\KeywordTok{geom_density}\NormalTok{(}\DataTypeTok{colour=}\StringTok{"pink"}\NormalTok{, }\DataTypeTok{size=}\DecValTok{1}\NormalTok{) }\OperatorTok{+}
\StringTok{  }\KeywordTok{geom_vline}\NormalTok{(}\DataTypeTok{xintercept =}\NormalTok{ mean1000, }\DataTypeTok{size=}\DecValTok{1}\NormalTok{, }\DataTypeTok{colour=}\StringTok{"#0000CC"}\NormalTok{) }\OperatorTok{+}\StringTok{ }
\StringTok{  }\KeywordTok{scale_x_continuous}\NormalTok{(}\DataTypeTok{breaks=}\KeywordTok{seq}\NormalTok{(th}\DecValTok{-3}\NormalTok{,th}\OperatorTok{+}\DecValTok{3}\NormalTok{,}\DecValTok{1}\NormalTok{), }\DataTypeTok{limits=}\KeywordTok{c}\NormalTok{(th}\DecValTok{-3}\NormalTok{,th}\OperatorTok{+}\DecValTok{3}\NormalTok{))  }\OperatorTok{+}\StringTok{ }\KeywordTok{ggtitle}\NormalTok{(}\StringTok{"The Histogram for simulation of Samples Means with n = 1000"}\NormalTok{) }\OperatorTok{+}\StringTok{ }\KeywordTok{scale_x_continuous}\NormalTok{(}\StringTok{"Sample Means"}\NormalTok{) }\OperatorTok{+}\StringTok{ }\KeywordTok{ylab}\NormalTok{(}\StringTok{"Density"}\NormalTok{)}
\end{Highlighting}
\end{Shaded}

\begin{verbatim}
## Warning: Ignoring unknown parameters: arg
\end{verbatim}

\begin{verbatim}
## Scale for 'x' is already present. Adding another scale for 'x', which will
## replace the existing scale.
\end{verbatim}

\includegraphics{CourseProject11_files/figure-latex/disb1.3-1.pdf}

\hypertarget{conclusion-and-inference}{%
\subsubsection{Conclusion and
Inference}\label{conclusion-and-inference}}

We know that a Normal Distribution follows a Bell-shaped curve, where
the red line represents calculated normal distribution, which is
comparable to the shape of histogram. The Central Limit Theorem states
that the sample means will become that of a standard normal distribution
as the sample size increases as well as meeting two conditions of
independence (n \textless{} 10\% ) and normal, or if skewed
distribution, then n \textgreater{} 30. Thus, from the graph we see, the
calculated distribution of means of random sapmled exponential
distributions, overlaps quite properly with the normal distribution with
the assumed value of lambda.

\end{document}
